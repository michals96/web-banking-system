\documentclass[a4paper,12pt]{article}

\usepackage{etoc}
\usepackage{blindtext}
\usepackage[utf8]{inputenc}
\usepackage[USenglish,british,american,australian,english]{babel}
\usepackage{helvet}
\usepackage{graphicx}
\usepackage{color}
\usepackage{geometry}
\newcommand\tab[1][1cm]{\hspace*{#1}}
\geometry{hmargin={2cm, 2cm}, height=10.0in}

\begin{document}
\thispagestyle{empty}
%% ------------------------ NAGLOWEK STRONY ---------------------------------
\includegraphics[height=37.5mm]{agh_nzw_a_en_1w_wbr_pms}\\
\rule{30mm}{0pt}{
{\large \textsf{Faculty of Physics and Applied Computer Science}}\\
\rule{\textwidth}{3pt}\\
\rule[2ex]
{\textwidth}{1pt}\\
\vspace{7ex}
\begin{center}
{\LARGE \bf \textsf{Engineering thesis}}\\
\vspace{13ex}
% --------------------------- IMIE I NAZWISKO -------------------------------
{\bf \Large \textsf{Michal Stefaniuk}}\\
\vspace{3ex}
{\small \sf major: {\bf \textsf{applied computer science}}}\\
\vspace{1.5ex}
%{\small \sf specialisation: {\bf \textsf{software engineering}}}\\
\vspace{10ex}
%% ------------------------ TYTUL PRACY --------------------------------------
{\bf \huge \textsf{Web application developement using WebToolkit
C++ on the example of a banking service}}\\
\vspace{14ex}
%% ------------------------ OPIEKUN PRACY ------------------------------------
{\Large \sf Supervisor: {\bf \textsf{Dr Grzegorz Gach}}}\\
\vspace{22ex}
{\large \bf \textsf{Cracow, June 2021}}
\end{center}
%% =====  STRONA TYTULOWA PRACY MAGISTERSKIEJKIEJ ====

\newpage

\begin{center}
	{\bf\large\textsf{Oświadczenie studenta}}\\
\end{center}

%% ===== PL TYL STRONY TYTULOWEJ PRACY MAGISTERSKIEJKIEJ ====
{Uprzedzony(-a) o odpowiedzialnosci karnej na podstawie art. 115 ust. 1 i 2 ustawy z dnia 4 lutego 1994 r. o prawie autorskim i prawach pokrewnych (t.j. Dz. U. z 2018 r. poz. 1191 z pozn. zm.): ,,Kto przywlaszcza sobie autorstwo albo wprowadza w blad co do autorstwa calosci lub czesci cudzego utworu albo artystycznego wykonania, podlega grzywnie, karze ograniczenia wolnosci albo pozbawienia wolnosci do lat 3. Tej samej karze podlega, kto rozpowszechnia bez podania nazwiska lub pseudonimu tworcy cudzy utwor w wersji oryginalnej albo w postaci opracowania, artystyczne wykonanie albo publicznie znieksztalca taki utwor, artystyczne wykonanie, fonogram, wideogram lub nadanie.", a takze uprzedzony(-a) o odpowiedzialnosci dyscyplinarnej na podstawie art. 307 ust. 1 ustawy z dnia 20 lipca 2018r. Prawo o szkolnictwie wyzszym i nauce (Dz. U. z 2018 r. poz. 1668 z pozn. zm.) ,,Student podlega odpowiedzialnosci dyscyplinarnej za naruszenie przepisow obowiazujacych w uczelni oraz za czyn uchybiajacy godnosci studenta.", oswiadczam, ze niniejsza prace dyplomowa wykonalem(-am) osobiscie i samodzielnie i nie korzystalem(-am) ze zrodel innych niz wymienione w pracy.

\bigskip

Jednoczesnie Uczelnia informuje, ze zgodnie z art. 15a ww. ustawy o prawie autorskim i prawach pokrewnych Uczelni przysluguje pierwszenstwo w opublikowaniu pracy dyplomowej studenta. Jezeli Uczelnia nie opublikowala pracy dyplomowej w terminie 6 miesiecy od dnia jej obrony, autor moze ja opublikowac, chyba ze praca jest czescia utworu zbiorowego. Ponadto Uczelnia jako podmiot, o ktorym mowa w art. 7 ust. 1 pkt 1 ustawy z dnia 20 lipca 2018 r. Prawo o szkolnictwie wyzszym i nauce (Dz. U. z 2018 r. poz. 1668 z po 'zn. zm.), moze korzystac bez wynagrodzenia i bez koniecznosci uzyskania zgody autora z utworu stworzonego przez studenta w wyniku wykonywania obowiazkow zwiazanych z odbywaniem studiow, udostepniac utwor ministrowi wlasciwemu do spraw szkolnictwa wyzszego i nauki oraz korzystac z utworow znajdujacych sie w prowadzonych przez niego bazach danych, w celu sprawdzania z wykorzystaniem systemu antyplagiatowego. Minister wlasciwy do spraw szkolnictwa wyzszego i nauki moze korzystac z prac dyplomowych znajdujacych sie w prowadzonych przez niego bazach danych w zakresie niezbednym do zapewnienia prawidlowego utrzymania i rozwoju tych baz oraz wspolpracujacych z nimi systemow informatycznych.}\\

\bigskip

\newpage
\tableofcontents
\newpage
\section*{Introduction}
\addcontentsline{toc}{section}{\protect\numberline{}Introduction}%
\tab {Software engineering is a very wide area of engineering which particularly concerns developing and maintaining programming products. A development process itself is a major challenge to all people involved, starting with developers creating the code, continuing with product owners who are managing teams and coordinating the work flow and ending-up with managers who are setting the direction of the whole process.

\bigskip
The main motivation to create this thesis was simply to present a development process that includes basic and nowadays necessary tools which are significantly helpful in such process.

\bigskip
To carry out the development and present all tools it was decided that an example application will be a C++ web application created with a GUI library in modern C++ called \textbf{Web Toolkit}. 

\bigskip 
Nowadays in a software engineering world there is a trend to migrate desktop applications to the internet, which has also impacted a lot GUI desktop libraries in decreasing their utilities. Foreground reason behind choosing C++ to create a web application, which is quite unusual, was to show the tremendous possibilities that this language still provides and to exhibit the capabilities of open source libraries shared among developers.

\bigskip
The following contents was divided into five chapters describing the development process and the application itself with bottom line two chapters summarizing thesis and listing literature.

\
}

\section{Projects assumptions}

\subsection{Application's blue-print}
\subsection{Programming environment}
\section{WebToolkit C++ library}
\subsection{Library introduction}
\subsubsection{Introduction to Wt}
\subsubsection{Introduction to Wt::Dbo}
\subsubsection{Introduction to Wt::Auth}
\subsection{Widgets gallery}
\subsection{Library overview}

\section{DevOps layer}
\subsection{Distributed version-control system}}
\subsubsection{GitHub}
\subsubsection{BitBucket}
\subsection{Proprietary issue tracking}
\subsubsection{JIRA Software}
\subsubsection{Confluence}
\subsection{Containers}
\subsubsection{Docker}
\subsubsection{Kubernetes}
\subsubsection{Containers vs OS-level virtualization}
\subsection{Automation server}
\subsubsection{GitHub CMake workflows}
\subsubsection{CircleCI}
\subsubsection{Jenkins}

\section{Implementation}
\subsection{Server side session}
\subsection{Logging panel}
\subsection{Database}
\subsection{User features}
\subsubsection{Transferring money}
\subsubsection{Checking user balance}
\subsection{Admin features}
\subsubsection{Listing all users}
\subsubsection{Access to server logs}
\subsection{Modern C++ features}

\section{Testing}
\subsection{Unit testing}
\subsection{Regression testing}
\section{Summary}
\section{Literature}
\newpage

\linespread{1.3}
\selectfont

\end{document}